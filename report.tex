\documentclass[14pt]{extreport}
\usepackage{geometry}
\usepackage[T2A]{fontenc}
\usepackage[utf8]{inputenc}
\usepackage[russian]{babel}
\usepackage {extsizes}
\usepackage{cmap} %кодировка pdf-документа
\usepackage{textcomp} %%текстовые символы
\usepackage{indentfirst}% корректировка отступов
\usepackage{amssymb} %математические символы
\usepackage{amsmath}% математические конструкции
\usepackage{amsfonts}
\usepackage{graphicx}% поддержка графики
\DeclareUnicodeCharacter{2212}{-}

\geometry{a4paper, left = 20mm, right = 20mm, top = 13mm, bottom = 15mm}
  

\title{\textbf{Отчёт по четвёртому практическому заданию курса "Автоматическое извлечение информации из текстов"} }
\author{Выполнил: Алексей Грищенко, 209 группа}
\date{}




\begin{document}
\maketitle

\section*{Постановка задачи}
\begin{enumerate}
    \item Использовать предобученную на основе fasttext модель word2veс, предрассчитанную на большом интернет-корпусе Common Crowl, для подсчета косинусных близостей слов в датасетах wordsim-similarity и wordsim-relatedness (см. ссылку в следующем разделе).
    \item Посчитать корреляцию полученных значений близости человеческими оценками из датасетов с помошью корреляции Спирмена.
    \item Сделать вывод на основании полученных значений корреляции
\end{enumerate}

\section*{Ресурсы используемые в практической работе}
Ссылка на практическую работу:\newline \textbf{https://github.com/grishchenkoalexey2004/fasttext\_word2vec\_usage}\newline
\indentДатасеты \textbf{wordsim-similarity} и \textbf{wordsim-relatedness} содержащие в себе пары слов и человеческие оценки их лексической близости можно найти по ссылке: \textbf{http://alfonseca.org/eng/research/wordsim353.html}\newline
\indentМодель word2vec под названием \textbf{crawl-300d-2M-subword}, обученную на основе fasttext можно найти по ссылке: \textbf{https://fasttext.cc/docs/en/english-vectors.html}\newline
\indent Для подсчёта косинусной близости используются функции \textbf{dot} и \textbf{norm} из библиотеки \textbf{numpy}, отвечающие соответственно за вычисление скалярного произведения и нормы векторов.\newline
\indent Коэффициент корреляции Спирмена считается с помощью функции \textbf{spearmanr} из модуля \textbf{numpy.stats}


\section*{Результаты}
Пары слов из фаилов wordsim\_similarity\_goldstandart.txt и \linebreak wordsim\_relatedness\_goldstandart.txt с посчитанной для них косинусной близостью можно найти в фаилах \textbf{results\_similarity.txt} и \textbf{results\_relatedness.txt}.

\indent Корреляция Спирмена между человеческими оценками из wordsim\_similarity
и wordsim\_relatedness и косинусными расстояниями равняется 0.835 и 0.64 соответственно.

\section*{Выводы}
Между человеческими оценками и косинусными расстояниями наблюдается сильная связь 
(согласно шкале Чеддока). Результаты говорят о том, что из сходства контекстов, в которых два слова, с большой вероятностью следует лексическое сходство слов и наоборот. 

























\end{document}
